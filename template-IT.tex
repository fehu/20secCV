%%%%%%%%%%%%%%%%%%%%%%%%%%%%%%%%%%%%%%%%%
% Twenty Seconds Resume/CV
% LaTeX Template
% Version 1.0 (14/7/16)
%
% Original author:
% Carmine Spagnuolo (cspagnuolo@unisa.it) with major modifications by
% Vel (vel@LaTeXTemplates.com) and Harsh (harsh.gadgil@gmail.com)
%
% Further modified by Dmitry K.
%
% License:
% The MIT License (see included LICENSE file)
%
%%%%%%%%%%%%%%%%%%%%%%%%%%%%%%%%%%%%%%%%%

%----------------------------------------------------------------------------------------
%	PACKAGES AND OTHER DOCUMENT CONFIGURATIONS
%----------------------------------------------------------------------------------------

\documentclass[letterpaper]{twentysecondcv-IT} % a4paper for A4

%----------------------------------------------------------------------------------------
%	 SKILLS
%----------------------------------------------------------------------------------------

\def\customCVsections{
	\languageSection

	\programmingSection
}

\languages{Fluent/English}
\programming{50/C, 50/C++}
					  {\underline{Databases}: SQL and no-SQL}

%----------------------------------------------------------------------------------------
%	 PERSONAL INFORMATION
%----------------------------------------------------------------------------------------
% If you don't need one or more of the below, just remove the content leaving the command, e.g. \cvnumberphone{}

\cvname{Your Name}
\cvjobtitle{ Job }

\cvbirth{01/01/1900}
\cvcitizen{Earth}
\cvmatrial{} % martial status

\cvnumberphone{(000) 000 0000} % Phone number
\cvskype{???}
\cvmail{email@mail.com}

\cvlinkedin{}
\cvgithub{}
\cvsite{} % Personal website


%----------------------------------------------------------------------------------------

\begin{document}

\makeprofile % Print the sidebar

%----------------------------------------------------------------------------------------
%	 EDUCATION
%----------------------------------------------------------------------------------------
\section{Education}

\begin{twenty} % Environment for a list with descriptions
	\twentyitem
    	{2015 - 2017}
        {}
        {MSc., Computer Science}
        {\href{http://www.uoguelph.ca/}{University of Guelph, Canada}}
        {}
        {GPA: 3.7/4.0, First Class}
	\twentyitem
    	{2009 - 2013}
		{}
        {BEng., Computer Engineering}
        {\href{http://www.unipune.ac.in/}{University of Pune, India}}
        {}
        {First Class with Distinction}
	%\twentyitem{<dates>}{<title>}{<organization>}{<location>}{<description>}
\end{twenty}

\section{Research}
\begin{twenty}
	\twentyitem
    	{2015 - 2017}
		{}
        {MSc. Candidate, Graduate Research Assistant}
        {\href{http://www.uoguelph.ca/}{University of Guelph}}
        {}
        {
       	\textbf{Thesis}: Data Integration from Multiple Historical Sources to Study Canadian Casualties of WWI
        {\begin{itemize}
        \item Proposed a scalable stepwise deterministic method to reliably integrate datasets without labeled data. The method performs comparably with a method that incorporates a Support Vector Machine
        \item Constructed a rich longitudinal dataset to enable comphrehensive time-series analyses about WWI Canadian society and military
        \item \textbf{Tools}: R, Python, scikit-learn, BeautifulSoup, pandas, matplotlib
		\end{itemize}}
        }
\end{twenty}

\section{Publications}
L. Antonie, H. Gadgil, G. Grewal, and K. Inwood, “Historical Data Integration - A Study of WWI Canadian Soldiers,” in 2016 IEEE 16th International Conference on Data Mining Workshops (ICDMW), pp. 186-193, IEEE, 2016.

%----------------------------------------------------------------------------------------
%	 EXPERIENCE
%----------------------------------------------------------------------------------------

\section{Experience}

\begin{twenty} % Environment for a list with descriptions
\twentyitem
    	{April 2017 -}
		{Present}
        {Data Engineer}
        {\href{http://www.bell.ca/}{Bell}}
        {}
        {}
        \\
	\twentyitem
    	{Sep 2015 -}
		{May 2016}
        {Co-founder \& Full Stack Developer}
        {\href{http://www.localxchange.ca/}{LocalXChange Inc.}}
        {}
        {
        {\begin{itemize}
        \item In a team of 2, raised \$8,000 in funding from The Hub incubator at the University of Guelph, to develop a prototype hyperlocal content platform, aimed at delivering local news and events from community organizations to local users in realtime (i.e., hyperlocal)
        \item In a team of 3, built hybrid mobile \& web apps with Ionic, Angular.js and MongoDB, surpassing 1,000 users within a month since launch
        \item Met with city officials, including the Mayor of Guelph, and university officials to discuss marketing \& business strategies for the platform
    \end{itemize}}
        }
    \\
    \twentyitem
   		{Sep 2015 -}
		{Dec 2016}
        {Graduate Teaching Assistant}
        {\href{http://www.uoguelph.ca}{University of Guelph}}
        {}
        {
        {\begin{itemize}
        \item TA for CIS*2430 (OOP), CIS*4150 (Software Reliability \& Testing) and CIS*3530 (Database Systems \& Concepts) courses
    \end{itemize}}
        }
     \\
     \twentyitem
   		{Dec 2013 -}
		{Apr 2015}
        {Test Automation Engineer}
        {\href{http://www.synechron.com/}{Synechron}}
        {}
        {
        \begin{itemize}
        \item Developed a \textit{Keyword Driven} and \textit{Behavior Driven} test automation framework for \href{https://www.microsoft.com/en-ca/dynamics/crm.aspx}{Microsoft Dynamics CRM}. Wrote an efficient recursive function to search multi-level frames, reducing development time by approximately two weeks. Won SPOT award {\includegraphics[scale=0.05]{img/trophy.png}}

        \textit{My work opened up a new position in the organization, enabling it to earn additional revenue of \$3,500 per month (estimated)}
        \item Demonstrated by proof of concept that rewriting a test automation framework for \href{https://www.microsoft.com/en-ca/dynamics/erp-ax-overview.aspx}{Microsoft Dynamics AX}, using an open source library (White) instead of a proprietary one (Coded UI), would enable the team to save \$4,000 annually by downgrading the prevailing edition of Microsoft Visual Studio from \textit{Premium} to \textit{Professional}
    \end{itemize}
    	}

	%\twentyitem{<dates>}{<title>}{<location>}{<description>}
\end{twenty}

\end{document}
